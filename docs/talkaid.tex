\documentclass{article}
\usepackage{graphicx} % Required for inserting images

\title{TalkAID}
\author{
    Anna Benedetta Salerno\\
    \and
    Michele D'Arienzo\\
    \and
    Raffaele Monti\\
    \and
    Luigi Petrillo\\
    \and
    Samuele Sparno\\
}
\date{23 Gennaio 2024}

\begin{document}

    \maketitle

    \section{Introduzione}

    Il progetto TALKAID rappresenta davvero un passo avanti significativo nel campo della riabilitazione e del supporto alle persone con disabilità del linguaggio. La possibilità di offrire trattamenti completamente a distanza e in maniera asincrona è un'innovazione che potrebbe aprire nuove opportunità per un numero ancora maggiore di individui affetti da queste patologie.

    I metodi tradizionali potrebbero non essere altamente personalizzati alle esigenze specifiche dei pazienti, poiché potrebbe essere difficile adattarli in modo rapido ed efficiente.

    La registrazione e il monitoraggio dei progressi dei pazienti potrebbero essere complessa e limitata a causa della mancanza di strumenti tecnologici dedicati.

    La componente di Intelligenza Artificiale è ritenuta fondamentale ai nostri obiettivi siccome aggiunge un livello di personalizzazione e adattabilità, permettendo ai pazienti di ricevere esercizi mirati in base al loro grado di severità della patologia. Questo non solo rende il trattamento più efficace, ma anche più agevole per i logopedisti, in quanto saranno aiutati dall'IA nella scelta degli esercizi migliori. I pazienti verranno incoraggiati a perseguire con impegno il percorso di miglioramento attraverso le loro statistiche.

    \section{Obiettivi}

    Lo scopo del nostro progetto è quello di realizzare un agente intelligente che possa:

    • Consigliare un insieme di esercizi mirato per paziente, basandosi sull'esperienza del paziente o sul lasso di tempo passato dall'ultima interazione con quella tipologia di esercizio;

    • Migliorare il sistema di consigli nel tempo, facendolo evolvere sulla base dei feedback del logopedista, il quale deciderà se un esercizio è appropriato o meno;

    \section{Modello PEAS}

    • Performance:

    Le prestazioni dell’agente sono valutate attraverso le seguenti misure:

    la sua capacità di consigliare esercizi più il mirati possibile per il paziente in base alla patologia

    il punteggio dell'esercizio che il paziente ha effettuato su quell'esercizio

    quanto tempo è passato dall'ultima volta che il paziente ha svolto l'esercizio

    • Environment:

    L’ambiente è:

    Completamente osservabile, in quanto si ha accesso a tutte le informazioni relative ai pazienti, in particolare le patologie e gli esercizi svolti, e alla lista degli esercizi svolti per ogni paziente;

    Non deterministico, in quanto lo stato dell’ambiente cambia indipendentemente dalle azioni dell’agente;

    Sequenziale, in quanto le esercitazioni effettuate dai pazienti e le scelte del logopedista influenzano le decisioni future dell’agente;

    Dinamico, in quanto nel corso delle elaborazioni dell’agente, un paziente potrebbe svolgere un esercizio, cambiando in tal modo le sue esigenze;

    Discreto, Il numero di percezioni dell’agente è limitato in quanto ha un numero discreto di patologie, esercizi, pazienti, azioni e percezioni possibili;

    A singolo agente, in quanto l’unico agente che opera in questo ambiente è quello in oggetto.

    • Actuators:

    Pagina web dove viene creata una lista di esercizi in maniera tabellare consigliata per ogni paziente.

    • Sensors:

    Gli esercizi svolti dal paziente, gli esercizi assegnati al paziente da parte del logopedista, il punteggio per esercizio del paziente, e il tempo passato dall'ultima volta che il paziente ha effettuato l'esercizio.

    Il dataset è un riadattamento del database utilizzato dal sito web principale, eliminando gli attributi non necessari e aggiungendo gli attributi gravità di lettura e scrittura, necessari al fine di poter avere una più completa visualizzazione delle necessità reali del paziente

\end{document}
